\chapter{Future Work}
\label{ch:future}


\section{Vision of a WPT System using TR}
\label{sec:future-roadmap}
This research represents a first step in the exploration of building a
time reversal WPT system.
%
We demonstrate one possible realization of this idea in Figure~\ref{fig:SysImage}.


The proposed system consists of two basic components.
%
The first is a rectenna embedded in a client device.
%
The system as described in Section~\ref{sec:ltr-meth} would require an
out-of-band feedback channel between the receiver and transmitter. However, we
have demonstrated in Section~\ref{sec:selective-sim} that a transmitter can target
receivers entirely in-band.
%
Our system in Figure~\ref{fig:SysImage} builds on these findings.
%
The second system component is a charging station that performs the time reversal process.
%
This component is responsible for recording characteristic signals from the
receiver(s), time reversing the signals, and re-broadcasting them into the
environment.



In a practical system, the rectenna will be integrated into the hardware of a
mobile device, or into an external component that plugs into the battery.
%
The charging station would be connected to an external power source, but
could otherwise be located anywhere in the room. This system will be most effective in a low-loss scattering environment.

\begin{figure}[t]
\centering
\includegraphics[width=\columnwidth]{figs/future/WPTSys}
\caption[Proposed Time Reversal System]{Proposed time reversal WPT system. The charging station collects a sona that contains  spatial information unique to the receiver's location (1). The sona is time reversed (2), amplified, and broadcast back into the environment. The amplified signal reconstructs on the client device (3) and is converted to usable DC power by the rectifier (4). A small fraction of the signal is used to re-broadcast a new characteristic signal (5) into the environment, which will be collected in the next sona, creating a stable loop (1). This loop can be initiatied by a pulse from the charging station, and requires no input energy from the client device.}
\label{fig:SysImage}
\end{figure}

We have demonstrated the underlying concepts of the WPT system depicted in Figure~\ref{fig:SysImage}.
However, much work is necessary to transform this proof-of-concept technology to a functional product that consumers may use.
Future work discussed here falls into one of two categories. The first category includes established TR techniques that may be extended into the field of TR WPT.
The second category includes novel designs proposed by the team applicable only to TR WPT.

\section{Applying other techniques from TR to WPT}
\label{sec:future-tr}
Imaging of objects and non-invasive surgery are already common applications of TR.
As a result, many techniques have been developed to improve reconstruction quality on a target.
We suspect that many of these techniques may have applicability to our proposed system, but did not
have the opportunity to investigate them in our research.
However, TR methods are generally designed for targeting stationary objects, in unchanging environments. These methods may have novel advantages and disadvantages for TR WPT applications. We suggest that future research be dedicated to investigating the applicability of techniques such as iterative time reversal and exponential amplification to a practical WPT scheme.

\subsection{Iterations}

It has been well proven in the literature that time reversal focusing on an object can be repeated to improve waveform collapse on a target.~\cite{prada_iterative_1991} However, this technique has not yet been applied to TR WPT.

We suspect that iterative TR can improve the reconstruction of a signal over time. The side lobes in the time domain are representative of energy that may not be able to be rectified unless they are combined into the main peak. We predict that we can maximize our efficiency and reconstruction quality using this method. Additionally, we predict that this method will only be useful for stationary targets. Because iterating the time reversal process takes several more steps, the length of time required to focus on a target is increased considerably. Since the technique for focusing upon a moving receiver proposed earlier relies upon repeating the time reversal process many times, it may not be acceptable for the time required to perform TR to increase several hundred percent.

Both of these characteristics can be easily tested, using algorithms already applied to acoustic TR. An experimental setup very similar to ours could be applied to this research. A faster sona refresh rate would be required from the TR system for this method to be able to outpace the natural ``decay'' of the testing environment. Once an iterative algorithm has been tested and found successful it should be tested in a less homogeneously reflective environment, to determine it's performance in lossy environments.

The novelty of such research is primarily in the characteristics measured. The demonstration (or lack thereof) of filtering of lossy paths would also be a major finding. Models should be generated relating the sona refresh rate to performance gains using iterative method. If possible, multiple environments should be tested to generalize these results.

\subsection{Multiple Transmitters}

All of the work in this thesis was conducted using a single transmitting antenna. The incorporation of multiple antennas would haved required major modifications to our experimental setup, and thus was beyond the scope of our study. However, previous work in time reversal in other fields, such as acoustics, have demonstrated the ability to improve signal quality through the use of multiple transmitters.

In one study, Mathias Fink demonstrated that a single-channel time reversal mirror acts like a spatiotemporal matched filter, where the time reveresal process is a convolution of the injected signal with a time reversed ``image'' of that signal. Using a 96-element antenna array, Fink showed that using multiple channels created a stronger collapse at the reconstruction time and that it tended to cancel the contributions of the temporal side-lobes, ultimately providing a better peak-to-noise ratio. When the number of channels is sufficient, the result is an inverse filter, which should produce the ideal reconstruction~\cite{fink-multi-channel}.

Although a practical wireless power transfer system will likely not be able to reach this theoretical limit, the positive relationship between number of elements and reconstruction strength should hold: namely, that adding $N$ elements to the time reversal mirror should increase the reconstruction strength by \textit{at least} a factor of $N$. In addition, it should reduce the accidental creation of ``hot spots'' at locations other than the intended reconstruction point, which would be a necessary safety concern to address in a practical system.

\subsection{Exponential Amplification}

Exponential amplification can be used to counteract decreased reconstruction quality caused by system losses~\cite{bini-thesis}. Exponential sona amplification is a nonlinear amplification to the sona across its timespan. It is meant to counteract the larger losses (due to larger number of reflections) experienced by reflective orbits with larger paths.

However, it should be understood that this method only compensates for the losses experienced by a time reversed signal. It does not avoid losses altogether, and in fact more energy is lost through this method. As a result, this method sacrifices efficiency for reconstruction quality. This tradeoff could be significant to designers of a TR WPT system. It would be useful to quantify the effects of this tradeoff in a WPT context so that informed decisions can be made in the design of future systems.

\section{Novel Advancements for TR in a WPT context}
\label{sec:future-wpt}

There are aspects of a TR WPT system that create avenues for research and innovation. Many of these take advantage of the unique operating principles with a TR WPT system. Most TR applications care little about the environment, focusing only on the direct link between transmitter and receiver. In TR WPT the opposite is true. Most TR applications consider stationary - or near stationary - targets. TR WPT system must always consider moving targets, or at least stationary targets within a dynamic environment.

These and other considerations set TR WPT apart from other TR research, and require the development of novel techniques. Below we suggest several research topics specific to TR WPT, experiments that could explore them, and how the results of these experiments can benefit TR WPT.

\subsection{Sub Cavity}

It is clear to the team that a TR WPT system will likely need some environmental modifications to achieve an efficiency sufficient to see practical application. These modifications should also be designed to be as cheap and simple to install as possible, to ensure that system price remains practical.

We suggest a method of improving efficiency of TR wireless power transfer by introducing a ``sub cavity'' within the transfer environment. This ``sub cavity'' will be defined as a region of the environment that allows extremely efficient TR WPT. The sub cavity becomes much less lossy than the environment as a whole. TR paths will preferentially move through the sub cavity, and can dramatically improve the efficiency of the entire system. A successful sub cavity must have the following characteristics:

\begin{itemize}
  \item High transfer efficiency/low reflective losses
  \item Chaotic geometry to facilitate the function of TR
  \item Presence of one or more paths between the sub cavity and main environment
\end{itemize}

These qualities already exist in many environments. Many indoor environments contain metallic objects, either as part of the environment itself, or embedded within the walls of an environment. Most of these objects tend to be within a meter of ground level. Thus, the lowest meter of a room will behave as a sub-cavity in practical applications.

Experiments should be done to represent how much benefit these objects will give to a TR WPT system. First, efficiency should be measured in completely empty rooms composed of typical materials. Then individual items hypothesized to affect (both positively and negatively) TR WPT efficiency should be introduced individually. The position and arrangement of these items should be altered, and their impact on efficiency quantified. Finally, the position and arrangement of groups of items should be altered as well.

We hypothesize:
\begin{itemize}
  \item The position of an individual object becomes a negligible influence when it is more than ~0.5 meters from a charging station or client
  \item Quantity of reflective objects is more important than their size (surface area)
  \item Large numbers of reflective objects decreases the impact of loss elsewhere in the environment
\end{itemize}

If this experiment demonstrates an unacceptably low transfer efficiency in an ordinary environment, the creation of an artificial sub cavity may be necessary.

A proposal for an artificial sub cavity can be found in Figure~\ref{fig:subCav}. This cavity is a thin sheet of material that would be mounted to the ceiling of the room in question. The material in this sheet should have a smaller index of refraction than air; as a result any signal broadcast into can become ``trapped'' in the layer. At any given point along the barrier of the sub cavity only small amounts of signal will return to the main environment. A TR WPT system can selectively choose paths through the sub cavity that will focus on a given receiver with high transfer efficiency. Reflectors can be added to the sub cavity as needed to ensure that its complexity is enough to allow TR to occur.

\begin{figure}[h]
\centering
\includegraphics[width=\columnwidth]{future/subCavity}
\caption[Proposed ``Sub Cavity Design'']{A proposed sub cavity design. A transmitter (1) broadcasts an interrogation pulse into the environment. When the pulse reaches the low index of refraction material of the sub cavity, it refracts, taking on a lower angle (2). The interrogation pulse totally internally reflects within the cavity (3), until returning to the main environment. Some aspects of the signal reach the receiver, allowing a TR link to be established (4). A focal aspect of the geometry above the transmitter (5) is highlighted as a way of improving transmission.}
\label{fig:subCav}
\end{figure}

Research would focus on, firstly, proving that sub cavities can work for the purposes of increasing efficiency of transfer. Once this has been established, the geometry of the sub cavity should be optimized to work in a wide range of potential environments. Uniformity of signal coverage is a major concern, and must be tested either experimentally, or through use of simulation software.

\subsection{Protection of Information}
Many of the improvements suggested above are focused on increasing the efficiency of the final system. Most of these improvements assume that while the system may be lossy, it is enclosed. That is, while signals may be absorbed by the environment, they are not lost completely. However, complete loss can be expected to occur often in practical applications. This can be due either due to holes in the environment (such as windows or doors) or heavily absorptive materials. The use of a selective transmission channel (as proposed above) may mitigate the effects of a channel through which a signal may escape. However, other methods should also be considered.

Earlier our concern with losses focused primarily on their impact on efficiency. Information transfer was sufficient to allow TR to occur. However, complete losses remove information from the system, and severely damage the ability for TR to reconstruct on a target. This can be visualized using the idea of loss of outgoing paths. Figure~\ref{fig:outgo} shows how time reversal can be impacted by environments with complete loss.

\begin{figure}[h]
\centering
\includegraphics[width=\columnwidth]{future/outgoing}
\caption[Example of information loss on time reversal efficacy]{A representation of how complete losses affect time reversal. In this scenario, a transmitter (1) sends a signal in all directions in an environment with only one reflective surface. As can be seen, only a small percentage of the energy successfully reaches the receiver (2). If time reversal from (2) to (1) were to be attempted, a small and spatially unbalanced amount of energy would arrive at (1), resulting in a poor reconstruction.}
\label{fig:outgo}
\end{figure}

\newpage

This is partly why a large number of reflective surfaces is beneficial in a time reversal environment. However, we believe that this effect is more advantageous (from an information protection point of view) when it occurs near to the target of the time reversal process. This concept is discussed in Figure~\ref{fig:infoProtection}. Note that there is an issue with how much of tine incoming signal may be reflected on approach - we do not currently have a solution to this problem. Using an array of transmitters to increase the number of possible short orbit paths to the receiver should also mitigate the effects of information loss.

\begin{figure}[h]
\centering
\includegraphics[width=0.85\textwidth]{future/informationProtection}
\caption[Proposed TR information protection system]{A proposed method of protecting a TR system from loss of information. The ``target'' of the TR process (1) is surrounded by a surface of semi-reflective material. This surface reflects effectively ``splits'' the outgoing wave into several smaller pieces (2). The result is that a greater solid angle is preserved coming out of the antenna/reflector system in cases such as the one in Figure~\ref{fig:outgo}.}
\label{fig:infoProtection}
\end{figure}

We believe that this concept should be tested using simulation methods as a proof-of-concept. Should it function as expected, we propose moving forward with some sort of experimental demonstration of information preservation effects. Finally, a model should be created describing the effectiveness of the method. Particular attention should be paid to the size of the design - we expect this method to be difficult to miniturize.

\subsection{TR WPT applied to Internet of Things Devices}
Previously we have primarily considered how TR WPT could be applied to mobile devices. Mobile devices are ideal for TR WPT because they are generally used in enclosed areas, use relatively low power, and benefit from the removal of wires. However, there are other applications that may also benefit from the system proposed.

In particular we would like to consider how TR WPT could be applied to to internet of things devices. Internet of things depends on a large number of interlinked computers managing consumer and industrial machines. Internet of things devices are useful in automating simple tasks, and for collecting environmental data. These devices are dependent on continual connection to the internet; as a result their power demands tend to require them to be tethered to wall sockets, or often-replaced batteries. However, their power demand is low, and could be addressed using TR WPT even easier than mobile devices.

We propose that these devices could serve as the inspriation for an experiment into practical TR WPT. In particular, these devices give an opportunity to experiment how selective targeting can be used with groups of devices. In Section \ref{sec:selective-sim} we demonstrate how selective targeting can be done. However, applying this technique to groups of devices will require a more sophisticated technique than demonstrated here, and will serve as a novel extension of this research. Such an experiment will be an important first step in quantifying the efficiency of a TR WPT system in a practical scenario.
