\chapter{Conclusion}

\label{ch:conclusion}

In summary, here is a review of our collected investigations.

\subsection{Equipment and Expertise Limitations}


Concerning spatial profiling experiments, the trials conducted by TESLA suggest that the region of excitation around a target could be relatively broad compared to the size of our antenna. Following this, the team wanted to see whether it was possible to take advantage of this result through repeated re-targeting of an antenna while it moved away from the region of greater power. It was hoped that a positive result  would confirm that a TR WPT system could maintain steady power delivery to a moving target. Our results appear suggest that it should be, though TESLA was somewhat limited by the time our TRM took to aquire a new sona and retarget the antenna.

TESLA's research also investigated how antenna design and form could relate to TR convergence.This path yielded inconclusive results for a variety of reasons. Due to lack of resources and relative inexperience, the team approached antenna design and testing using a ``rapid prototyping'' plan that resulted in inconsistent designs. Higher quality fabrication techniques, coupled with prototypes in computer simulations, could have improved this aspect of research.  We suggest that future research in TR antenna optimization focus on the creation of nonlinearly reflective antennas useful in NLTR with consistent quality.  These antennas will allow experiments on TR WPT with multiple receivers, a topic which may offer significant advances in the field.

To better investigate TR's applicability to WPT, the team attempted to map how broad and how strong the excitation a collapsing wavefront was around a target antenna. An understanding of this would be useful to future engineers attempting to integrate a TR WPT system into any larger system. Reconstruction were mapped at a variety of frequencies, and the width of this spatial profile was found to be directly related to the wavelength.

On the topic of overlapping sonas, we hypothesized that overlapping reconstructions could result in more effective power transmitted per duty cycle. Our experiments seemed to support this, with an increased number of reconstructions per cycle continuing to maintain their characteristic shape. However, one setback in our experimentation was the Agilent E8267D Vector PSG, which attempted to scale each arbitrary waveform  broadcast to the same total output power. This prevented us from definitively concluding a relationship between the number of reconstructions and the power received by the target.

Better tools can also help the team revisit previously unsuccessful experiments.The team investigated whether ferromagnetic nanorods could be used to produce a distinctive nonlinear signature, but these tests were inconclusive. TESLA could not identify the presence of a strong nonlinear signal at the scanned frequencies, though a different setup or frequency range may yield more conclusive results. 

As discussed in the literature review, prior research has distinguished between linear and nonlinear elements. We wanted to demonstrate that it was also possible to distinguish between two distinct nonlinear elements. While we were succesful in replicating previous work distinguishing linear and nonlinear elements, we were not able to produce two nonlinear elements that were sufficiently different and identifiable in their nonlinear response in frequency domain. However, we remain convinced that this is possible based on the results of our simulations.

Additionally, we attempted to make our own rectenna that would both rectify incoming microwaves to a DC voltage and act as a nonlinear element for NLTR. We were able to demonstrate both these capabilities in a test environment. However, significant optimization remains necessary before our rectenna is usable for full-scale NLTR-based WPT.


\subsection{Future Work and Considerations}

TESLA's research into TR was very much a learning experience.  The team began its research with little to no experience with TR.  As a result, many of the team's early research results should be reconsidered or revisited. There are other areas that the team did not investigate at all, and which now are obvious areas of research.

TESLA's experimentation focused significantly on the relationship between sona and reconstruction, and there are more areas within this topic that should be revisited in greater detail.  TESLA performed several early experiments on partial sonas to see how they relate to reconstruction quality.  However, the team prioritized completion of other tests at the time, as there was an unclear sense of practical application or benefit by continuing.  In retrospect, these tests may be useful in relating the sona to the environmental characteristics and should be considered in more detail.

Future researchers may want to consider the quality of reconstructions created with partial sonas.  This research should also give a better quantification of what ``reconstruction quality'' means for arbitrary reconstruction waveforms. It could be beneficial to find methods of relating sona length and general shape to geometric factors of the environment. Understanding this relationship can allow the optimization of a TR WPT system for a given environment. The reverse is also true; some modifications to the transmission environment may help improve efficiency.

Finally, the effect of iterative time reversal on reconstruction was ignored in TESLA's tests.  The team suggests that it should be considered in future experimentation. Iterative time reversal mirrors are common in the literature for signal focusing, and there is reason to believe that iteration will also improve EM TR WPT.  Iteration may have significant effects on transfer efficiency and spatial profile.  However, iteration is also based off of the idea of a stationary target, and may not work well for moving targets.  This tradeoff, once understood, will greatly help optimize the system.

The aluminum echoic chamber used in tests (the Gigabox), and the equipment used to transmit and measure EM waves within this chamber were from previous experiments on TR done by graduate students under Dr. Anlage.  These resources allowed TESLA a great deal of freedom in designing and conducting experiments.  However, monetary and physical resources of the team limited the purchase of additional equipment, which in turn restricted the experiments that could be done. 

The Gigabox was designed to be an ideal environment to study TR from a signal processing point of view. However, the Gigabox is not easily modifiable. A Gigabox with interchangeable panels will allow for the creation of testing environments of differing absorption, translucence, and geometry. Finer control of these parameters will aid the development of a model for the effects of environment on the power losses of TR WPT. The lack of this model for transmission efficiency is one of the greatest challenges towards designing a TR WPT system.

Thanks to the generosity of Dr. Anlage and the UMD CNAM, Team TESLA had access to several state-of-the-art measuring and testing devices. Researching alongside graduate students associated with the CNAM, TESLA explored TR WPT primarily through the behavior of reconstruction, including how sonas can be manipulated to change reconstruction parameters.  Experiments including minimum TR cycle time, overlapping sonas, and reconstruction profile characterization have added to the literature surrounding practical TR, but gaps in knowledge continue to remain.  Many of these gaps we have identified are due to limitations in TESLA's technical expertise and lack of  access to proper equipment.  As a result of these shortcomings, we suggest areas of further research that could yield fruitful results with a dedicated and pioneering team, including what experiments could address these gaps.  It is our hope that our research will lay groundwork for further explorations of how TR can be applied to practical WPT.
