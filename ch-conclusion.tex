\chapter{Conclusion}

\label{ch:conclusion}

\todo{Need to fill this out}

\section{Future Work}

\todo{Consider looking at last paragraph or two from Scott’s section.}

Thanks to the generosity of Dr. Anlage and the UMD CNAM, Team TESLA had access to several state-of-the-art measuring and testing devices. Researching alongside graduate students associated with the CNAM, TESLA explored TR WPT primarily through the behavior of reconstruction, including how sonas can be manipulated to change reconstruction parameters.  Experiments including minimum TR cycle time, overlapping sonas, and reconstruction profile characterization have added to the literature surrounding practical TR, but gaps in knowledge continue to remain.  Many of these gaps we have identified are due to limitations in TESLA's technical expertise and lack of  access to proper equipment.  As a result of these shortcomings, we suggest areas of further research that could yield fruitful results with a dedicated and pioneering team, including what experiments could address these gaps.  It is our hope that our research will lay groundwork for further explorations of how TR can be applied to practical WPT.

\subsection{Equipment Limitations}

The aluminum echoic chamber used in tests (the Gigabox), and the equipment used to transmit and measure EM waves within this chamber, were from previous experiments on TR done by graduate students under Dr. Anlage.  These resources allowed TESLA a great deal of freedom in designing experiments.  However, monetary and physical resources of the team limited the purchase of additional equipment, and this in turn restricted the experiments that could be done.

The Gigabox was designed to be an ideal environment to study TR from a signal processing point of view. However, the Gigabox is not easily modified from this purpose. A Gigabox with interchangeable panels will allow for the creation of testing environments of differing absorption, translucence, and geometry. Fine control of these parameters will aid the development of a model for the effects of environment on the power losses of TR WPT. The lack of this model for transmission efficiency is one of the greatest hurdles towards designing a TR WPT system.

Better tools can help the team revisit previously unsuccessful experiments. Early in TESLA’s research the team investigated how antenna design and form could relate to TR convergence.  This path yielded inconclusive results for a variety of reasons. Due to lack of resources and inexperience, the team approached antenna design and testing using a “rapid prototyping” plan that resulted in inconsistent designs. Higher quality fabrication techniques, coupled with prototyping in simulations, could have improved this aspect of research.  We suggest that future research in TR antenna optimization should focus on the creation of nonlinearly reflective antennas useful in NLTR.  These antennas will allow experiments on TR WPT with multiple receivers, a topic which may offer significant advances in the field.

\subsection{Technical Inexperience}

TESLA’s experimentation focused significantly on the relationship between sona and reconstruction.  However, there are more areas within this topic that should be revisited in greater detail.  TESLA performed several early experiments on partial sonas to see how they relate to reconstruction quality.  However, the team prioritized completion of other tests at the time, as there was an unclear sense of practical application or benefit by continuing.  In retrospect, these tests may be useful in relating the sona to the environmental characteristics and should be considered in more detail.

Future researchers may want to consider the quality of reconstructions created with partial sonas.  This research should also give a better quantification of what “reconstruction quality” means for arbitrary reconstruction waveforms. It could be beneficial to find methods of relating sona length and general shape to geometric factors of the environment. Understanding this relationship can allow the optimization of a TR WPT system for a given environment. The reverse is also true; some modifications to the transmission environment may help improve efficiency.

Finally, the effect of iterative time reversal on reconstruction should be considered.  Iterative time reversal mirrors are common in the literature for signal focusing [CITE HERE], and there is reason to believe that iteration will also improve EM TR WPT.  Iteration may have significant effects on transfer efficiency and spatial profile.  However, iteration is also based off of the idea of a stationary target, and may not work well for moving targets.  This tradeoff, once understood, will greatly help optimize the system.

\todo{Need to fill this out...}


\subsection{Summary}

Outline:
	What discoveries or accomplishments have been made:
Overlapping Reconstructions
Ability to produce continual reconstructions on a stationary target
Simultaneous reconstructions {linear TR}?
Spatial Profile
Description of localization of reconstructions on a stationary targets
Moving Reconstructions
Demonstrate capability to follow moving receiver based on spatial profile of reconstruction on a quasi-stationary receiver.
Selective Targeting
Ability to selectively target diodes by exploiting their separate nonlinear characteristics.
Simultaneous reconstructions (non via overlapping sonas {nonlinear TR})?
ferromagnetic nanorods
Identifying potential nonlinear elements to be used in NLTR
Properties of nanorod harmonic generation: amplification, suppression, and/or absence of nonlinear response
Rectenna

In summary, several technical accomplishments have been made to assist
