\chapter{Conclusion}

\label{ch:conclusion}

\todo{Need to fill this out}

\subsection{Future Work}

Consider looking at last paragraph or two from Scott’s section.

Due to the generosity of Dr. Anlage and CNAM, Team TESLA had access to several state-of-the-art measurement and testing devices. The aluminum echoic chamber used in tests (the Gigabox), and the equipment used to transmit and measure EM waves within this chamber, were from previous experiments on TR done by graduate students under Dr. Anlage.  These resources allowed TESLA a great deal of freedom in designing experiments.  However, monetary and physical resources of the team limited the purchase of additional equipment, and this in turn restricted the experimentation that could be done.

The Gigabox is an ideal environment to study TR from a signal processing point of view, which is what previous research focused on. However, the Gigabox however is not easily modified, which prevents experimentation on the effects of environment on the power losses of TR WPT.  We propose a redesigned Gigabox with the ability to easily add and remove panels.  This will allow for easy and comprehensive testing environments of differing absorption, translucence, and geometry.

The lack of a model for transmission efficiency is currently the greatest hurdle towards designing a TR mirror that can achieve effective WPT.

---

	Early in TESLA’s research the team investigated how antenna design and form could relate to TR convergence.  This path yielded inconclusive results for a variety of reasons. The most pressing of these was the extremely limited experience by the team in the field of antenna design and characterization, which hampered efforts to generate conclusive results.  In addition, the team approached antenna design and testing using a “rapid prototyping” plan that resulted in inconsistent designs.  Higher quality fabrication techniques such as machining and printing, coupled with prototyping in simulations, could have improved this aspect of research.
	Future research in TR antenna optimization should focus on the creation of nonlinearly reflective antennas useful in NLTR.  These antennas will allow experiments on TR WPT with multiple receivers, a topic which may offer significant advances in the field.

----

	TESLA’s experimentation focused significantly on the relationship between sona and reconstruction.  However, there are more areas within this topic that should be revisited in greater detail.  TESLA performed several early experiments on partial sonas to see how they relate to reconstruction quality.  However, the team prioritized completion of other tests at the time, as there was an unclear sense of practical application or benefit by continuing.  In retrospect, these tests may be useful in relating the sona to the environmental characteristics and should be considered in more detail.
	Future researchers may want to consider the quality of reconstructions created with partial sonas.  This research should also give a better quantification of what “reconstruction quality” means for arbitrary reconstruction waveforms.  Methods of relating sona length and general shape to geometric factors of the environment.

	Finally, the effect of iterative time reversal on reconstruction should be considered.  Iterative time reversal mirrors are common in the literature for signal focusing [CITE HERE], and there is reason to believe that iteration will also improve EM TR WPT.  Iteration may have significant effects on transfer efficiency and spatial profile.  However, iteration is also based off of the idea of the stationary target.  That said, 

---

TESLA’s research has primarily explored TR WPT primarily through the behavior of the reconstruction, and how sonas can be manipulated to change reconstruction parameters.  Experiments such as minimum TR cycle time, overlapping sonas, and reconstruction profile characterization expand the understanding of practical TR behavior, but many knowledge gaps remain.  Many of these gaps are due to limitations in the team’s technical experience, materials, or time.  Some research areas that the team sees as potentially fruitful are outlined below.  Experiments that could address these gaps are also proposed.  It is our hope that our research can lay the groundwork for further explorations of how TR can be applied to practical WPT.

\todo{Need to fill this out...}


\subsection{Summary}

Outline:
	What discoveries or accomplishments have been made:
Overlapping Reconstructions
Ability to produce continual reconstructions on a stationary target
Simultaneous reconstructions {linear TR}?
Spatial Profile
Description of localization of reconstructions on a stationary targets
Moving Reconstructions
Demonstrate capability to follow moving receiver based on spatial profile of reconstruction on a quasi-stationary receiver.
Selective Targeting
Ability to selectively target diodes by exploiting their separate nonlinear characteristics.
Simultaneous reconstructions (non via overlapping sonas {nonlinear TR})?
ferromagnetic nanorods
Identifying potential nonlinear elements to be used in NLTR
Properties of nanorod harmonic generation: amplification, suppression, and/or absence of nonlinear response
Rectenna

In summary, several technical accomplishments have been made to assist
