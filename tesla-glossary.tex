%% GLOSSARY ENTRIES

\newglossaryentry{time-reversal}
{
    name=time reversal,
    description={Time reversal can refer to $(1)$ the property of time reversal symmetry in the system; or $(2)$ the signal focusing technique using said property. \(In this thesis, the abreviation TR will refer to \(2\)\)}
}
\newglossaryentry{pulse-inversion}
{
   name=pulse inversion,
  description={The signal processing technique for isolating the nonlinear component of a signal.}
}
\newglossaryentry{efficiency}
{
   name=efficiency,
  description={The ratio of usable energy delivered to the total energy broadcast by a WPT system. How efficiency is measured can vary wildly based on methodology. In this paper we will define efficiency as being the total power delivered to a load, compared to the total power drawn by the transmitter}
}
\newglossaryentry{range}
{
   name=range,
  description={The maximum distance that a wireless power transfer system can transmit a given amount of power.}
}
\newglossaryentry{max-power}
{
   name=maximum power,
  description={The maximum amount of power that can be transmitted to a target using a given wireless power transfer system.}
}
\newglossaryentry{active-transmission}
{
   name=active transmission,
  description={A WPT system where both transmitter and receiver expend some power to maintain the process.}
}
\newglossaryentry{passive-transmission}
{
   name=passive transmission,
  description={A WPT system where only the transmitter expends power to maintain the process.}
}
\newglossaryentry{transmitter}
{
   name=transmitter,
  description={The component of a WPT system that outputs power into the system.}
}
\newglossaryentry{receiver}
{
   name=receiver,
  description={The component of a WPT system that has a net input of power into the system.}
}
\newglossaryentry{impedance-matching}
{
   name=impedance matching,
  description={Adapting the impedance of a load to the impedance of a power source, such that power delivered to the load is maximized.}
}
\newglossaryentry{sona}
{
   name=sona,
  description={Sona: originating from sonabilis, resonant. The signal recovered at the receiver during the interrogation stage of a TR process. During a TR process the sona is time reversed and injected into the cavity to form a reconstruction at a receiver.}
}
\newglossaryentry{reconstruction}
{
   name=reconstruction,
  description={A focused signal produced as the result of a TR process. Reconstuctions are formed from the injection of time-reversed sona at the transmitter.}
}
\newglossaryentry{collapse}
{
   name=collapse,
  description={The movement of a time reversed sona towards its point of convergence.  The collapse occurs as the reflections of the sona interfere constructively at the point of convergence}
}
\newglossaryentry{scatterer}
{
   name=scatterer,
  description={An element that functions to isotropically scatter ambient waves. A reflector.}
}
\newglossaryentry{anisotropic-scatterer}
{
   name=anisotropic scatterer,
  description={A scattering element with a directional bias.}
}
\newglossaryentry{port}
{
   name=port,
  description={An hole on the Gigabox, used to transmit signals into the Gigabox, and record signals within it.}
}
\newglossaryentry{Gigabox}
{
   name=Gigabox,
  description={The microwave cavity used for our experimental TRM.}
}
\newglossaryentry{interrogation-pulse}
{
   name=interrogation pulse,
  description={The signal injected into the cavity at the transmitter in order to obtain a sona.}
}
\newglossaryentry{mode}
{
   name=mode,
  description={An eigenfunction of the wave equation given the boundary conditions of a system.}
}
\newglossaryentry{carrier-frequency}
{
   name=carrier frequency,
  description={The frequency of the carrier wave for our interrogation process.}
}
\newglossaryentry{linear-tr}
{
   name=linear time reversal,
  description={Time reversal involving a transmitter and receiver.  Requires some separate method of communication of sonas between transmitter and receiver.}
}
\newglossaryentry{nonlinear-tr}
{
   name=nonlinear time reversal,
  description={Time reversal involving a transmitter and nonlinear receiver.  During use the nonlinear receiver will generate nonlinear harmonics allowing the sona that it generates to be filtered out from background noise.}
}
\newglossaryentry{s-parameters}
{
   name=S parameters,
  description={The parameters of a scattering matrix which describe a linear electronic networks's transmission/reflection of an approaching EM wave.}
}
\newglossaryentry{injecting-port}
{
   name=injecting port,
  description={A port designated for broadcasting (injecting) signals. The transmitter of the TRM.}
}
\newglossaryentry{recording-port}
{
   name=recording port,
  description={A port designated for recording broadcasts injected into the cavity. The receiver of the TRM.}
}
\newglossaryentry{parasitic-capacitance}
{
   name=parasitic capacitance,
  description={The unwanted capacitance introduced from the arrangement of components in an electrical network along with materials.}
}
\newglossaryentry{knee-voltage}
{
   name=knee voltage,
  description={The voltage nescessary for ``turning on'' a diode.}
}
\newglossaryentry{mode-density}
{
   name=mode density,
  description={A discription of the seperation of modes in terms of frequency (energy) separation; as $\delta f = 0$ the mode spectrum tends to form a continuum.}
}
\newglossaryentry{ray-chaotic-environment}
{
   name=ray chaotic environment,
  description={An important feature of a chaotic system where there is a sensitive dependence on initial conditions; if one were to consider two ray originating at a given point with an arbitrarily small angle in separation, the forward ray traces quickly diverge. For our application, the implication is these ray traces will cover the volume of the cavity.}
}
\newglossaryentry{short-orbit}
{
   name=short-orbit,
  description={The mechanism responsible for the multiple reconstruction signals appearing at different time intervals. }
}
\newglossaryentry{rectenna}
{
   name=rectenna,
  description={A rectifying antenna. It collected electromagnetic power and converts it into usable DC power. This device is commonly used in wireless power transfer systems. }
}

%% ACRONYMS
\newacronym{wpt}{WPT}{Wireless Power Transfer}
\newacronym{tr}{TR}{Time Reversal}
\newacronym{nltr}{NLTR}{Nonlinear Time Reversal}
\newacronym{hrmc}{HRMC}{Highly Resonant Magnetic Coupling}
\newacronym{trm}{TRM}{Time Reversal Mirror}
\newacronym{dort}{DORT}{Decomposition of the Time Reversal Operator}
\newacronym{ltr}{LTR}{Linear Time Reversal}
\newacronym{awg}{AWG}{Arbitrary Waveform Generator}
\newacronym{psg}{PSG}{Pulse Signal Generator}
\newacronym{dso}{DSO}{Digital Storage Oscilloscope}
\newacronym{fft}{FFT}{Fast Fourier Transform}
\newacronym{cnam}{CNAM}{Center for Nanophysics and Advanced Materials}
\newacronym{em}{EM}{Electromagnetic}
\newacronym{cst}{CST}{Computer Simulation Technology: Microwave Studio}
\newacronym{fit}{FIT}{Finite Integration Technique}
\newacronym{fdtd}{FDTD}{Finite Difference Time Domain}
\newacronym{pwm}{PWM}{Pulse Width Modulation}
\newacronym{trr}{TRR}{Reverse Recovery Time}
\newacronym{rf}{RF}{Radio Frequency}
