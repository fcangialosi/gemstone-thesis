\chapter{Introduction}
\label{ch:introduction}

For over a century, civilization has become accustomed to simply plugging into a wall socket in order to power its devices. But what if the outlet could be taken out of the picture? What if these devices could operate beyond the confines of cord and cable? At present, this concept has more basis in science fiction; however, this fiction is quickly becoming reality. Although the concept of wireless electricity has existed since its proposal by Nikola Tesla in 1891, it was left largely uninvestigated throughout the twentieth century \cite{NikkiT}. In the past decade, significant strides have been made in this field, but the age of wireless power has only just begun, and there remain methods of transmitting this power to electronic devices still unexplored.

\subsection{Introduction to Wireless Power}
The ability to wirelessly transport energy from one point to another has the potential to attract many unique investors and customers. Such a technology is convenient; it saves space by eliminating the need for unwieldy chargers or elusive wall outlets. Businesses that offer Wi-Fi connectivity may be interested in this technology as another service to both retain and grow their customer bases. A convenient method of recharging devices will discourage the use of non-reusable batteries and present environmental benefits. Residential consumers would certainly be interested in a convenient way to charge their cell phones, laptops, and other small devices. Automobile companies such as Toyota, Infiniti, and Volvo have demonstrated interest in wireless power technology, so users can charge their electric or hybrid cars more conveniently. There are also surely technologies, which will develop as a consequence of wireless power transfer (WPT), that have not yet been fathomed.

One specific example in which wireless power could be particularly beneficial is an airport terminal, where individuals may wait for hours until their flight takes off. Many of these travelers will have a need to charge laptops, cellphones, and other electronic devices for both personal and work reasons. However, the layout of an airport terminal means that most people will be unable to reach a plug with their limited charging cords. Often, this creates a high amount of stress and unrest among travelers in the airport, particularly where individuals need their devices to get work done. This example can be extended anywhere there are more people who desire power for their devices, and the use of outlets is inconvenient. The college library is another paradigm in which hundreds of people are confined to a space where outlet access may be an issue.

These are not the only locations where individuals plug in their devices regularly. Wireless power technology could in fact have its most widespread impact in a domestic setting. The Federal Energy Regulatory Commission (FERC) notes that there are 80,000 GWh used per week on average across the entire United States [2]. Assuming even just 5\% of this power goes to powering laptops, cell phones, and tablets, this leaves 4,000 GWh of power per week that go to these devices. The topic of convenient power transmission clearly transcends just an airport or library. With so many people plugged into the grid, it is apparent that new methods of energy transfer would benefit millions of people in a multitude of situations. From domestic users to students and travelers, it is clear that millions of people will benefit from improved delivery of power.

\subsection{Our Research}
Team TESLA plans to examine a theoretical technique that has the potential to drastically improve the range of a WPT system and eliminate the issue of alignment. This technique, known as electromagnetic time reversal, has already been proven to be a successful method of information transfer [3][4], but its application to WPT has not yet been investigated. Time reversal (TR) is a method of locating the position of a transmitter by utilizing the reflection and interference of waveforms in a complex environment. This indicates that it should be an ideal method for WPT, as rooms and buildings contain complex surfaces, and TR can determine the location of the device’s battery without knowing any information about the environment itself.
The questions that TESLA intends to investigate are as follows: With what electrical efficiency and over which distances can power be transferred wirelessly using the method of time reversal? What impact does the environment have on these values? Finally, how is this efficiency affected by the motion of the receiver?
TESLA’s first hypothesis posits that significant amounts of power can be transmitted using TR. TESLA is defining a specific benchmark for this number to be at least one watt; it has already been proven that TR can transmit small packets of electricity in the form of data signals [3]. As such, it is believed that the same technique can be used for larger quantities of power because the amplitude of an electromagnetic wave does not impact factors such as frequency, wavelength, and dispersion. By leaving these factors untouched, TESLA believes that TR WPT can be scaled up using different circuitry setups. This conjecture is important to test, as the following hypotheses rely upon it.
The team’s second hypothesis proposes that TR methods will be on par with or more efficient – defined here as the ratio between the power received by the load and the power transmitted – than other WPT techniques. In traditional methods of WPT, signals are broadcasted in all directions, which results in significant energy losses. In a TR WPT system however, the energy will be focused on a single point. Although there will be some amount of loss due to absorption by the environment, it should be significantly less than the amount of loss from radiating power in all directions. Generally, TESLA believes that this method will optimize range and efficiency without compromising safety.
Finally, TESLA hypothesizes that the efficiency of the TR-based WPT will not decrease significantly if the receiver is moving in an arbitrary fashion relative to the transmitter. It is logical that, since electromagnetic waves traverse at the speed of light, any distance traveled by the receiver during the process will be so miniscule that it can be ignored [5]. During statistical analysis of collected data, TESLA seeks to reject the null hypothesis that TR is only a viable method for transmitting data, not power, and that the transmission will break down if the receiver is in motion.
