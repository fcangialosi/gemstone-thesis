\chapter{Introduction}
\label{ch:introduction}

For over a century, we have become accustomed to simply plugging into a wall socket in order to power its devices. But what if the outlet could be taken out of the picture? What if these devices could operate beyond the confines of cord and cable? Although the concept of wireless electricity has existed since its proposal by Nikola Tesla in 1891, it was left largely uninvestigated throughout the twentieth century ~\cite{NikkiT}. In the past decade, the proliferation of mobile devices and improved processing power have encouraged significant strides in the wireless power transfer (WPT) field. The age of wireless power has only just begun however, and there remain unexplored methods of transmitting this power to electronic devices.

\section{Introduction to Wireless Power}
The ability to wirelessly transport energy from one point to another has the potential to attract many unique investors and customers. Such a technology is convenient; it saves space by eliminating the need for unwieldy chargers or elusive wall outlets. Businesses that offer Wi-Fi connectivity may be interested in this technology as another service to both retain and grow their customer bases. A convenient method of recharging devices will discourage the use of non-reusable batteries and present environmental benefits. Residential consumers would certainly be interested in a convenient way to charge their cell phones, laptops, and other small devices. Automobile companies have demonstrated interest in wireless power technology, so users can charge their electric or hybrid cars more conveniently \cite{ToyotaNews}\cite{VolvoNews}.

The benefits of wireless power transfer (WPT) systems are myriad. Imagine an airport terminal, where individuals may wait for hours until their flight takes off. Many of these travelers will have a need to charge laptops, cellphones, and other electronic devices for both personal and work reasons. However, the layout of an airport terminal means that most people will be unable to reach a plug with their limited charging cords. Often, this creates a high amount of stress and unrest among travelers in the airport, particularly where individuals need their devices to get work done. This example can be extended anywhere there are more people who desire power for their devices, and the use of outlets is inconvenient. The college library is another paradigm in which hundreds of people are confined to a space where outlet access may be an issue. From domestic users to students and travelers, it is clear that millions of people will benefit from improved delivery of power.

\section{Our Research}
Team TESLA plans to examine a technique that has the potential to drastically improve the range of a WPT system and eliminate the issue of alignment. This technique, known as electromagnetic time reversal, has already been proven to be a successful method of information transfer ~\cite{nltr-wave-chaotic,cepni2005experimental}, but its application to WPT has not yet been investigated. Time reversal (TR) is a method of locating the position of a receiver by utilizing the reflection and interference of waveforms in a complex environment. Because of this, we believe that TR may be an ideal method for WPT. Rooms and buildings contain complex surfaces that will allow TR to determine the location of the device's battery without knowing any information about the environment itself.

TESLA has investigated the feasibility of applying TR to WPT. In particular, we have considered the following questions:
\begin{itemize}
\item Can energy be transmitted consistently using TR?
\item Can TR transmit energy to a moving target?
\item Can TR selectively transmit energy to different targets?
\end{itemize}

It has already been proven that TR can transmit small packets of energy in the form of data signals ~\cite{nltr-wave-chaotic}. However, the duration of the data signals was small compared to the spacing between them. As such, previous TR methods cannot be used to transmit energy as is.  However, this technique can be modified to transmit packets of energy more rapidly. Our first experiments have shown that energy can be regularly transmitted using TR.

We have also demonstrated that TR WPT can effectively transmit energy to a moving target. Previous work in electromagnetic TR has exclusively used stationary targets. Time reversed systems are dependent on the reflections in a system, and even small changes in geometry (Such as a change in target position) result in the method being useless.  TESLA has demonstrated that a rapidly updating time reversal mirror (TRM) can compensate for the movement of the target.  More importantly, we have characterized the behavior of such an updating system, allowing future TR WPT systems to be designed considering the result.

Finally, TESLA has explored the use of nonlinear time reversal (NLTR) to distinguish between targets in an environment with multiple possible targets. \cite{nltr-wave-chaotic} demonstrated that targets with nonlinear elements can be distinguished from linear targets.  TESLA has used simulation software to demonstrate that this concept can be extended to distinguish between different nonlinear targets, a concept that may be necessary in the practical application of a TR WPT system.
