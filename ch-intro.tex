\chapter{Introduction}
\label{ch:introduction}

In 1998, accessing the Internet required a clunky desktop computer and a dial-up subscription. When Apple integrated WiFi into its iBook computers under the brand name ``AirPort'' the following year, it forced other industry giants to compete and revolutionized wireless networking~\cite{_brief_2004}. If not for the adoption of this technology along with the advent of wireless data-enabled and commercially available cell phones such as Nokia's Communicator 9000~\cite{_first_1996}, cell phones would never have evolved to surpass their primary function as telephones and become the communication necessity they are today.

The ubiquity of WiFi and wireless communication revolutionized the way society interacts with the Internet. Suddenly, portable devices became powerful standalone devices rather than accessories to their desk-bound counterparts. Cell phones rivaled computers in necessity and ubiquity, and eclipsed their predecessors in terms of productivity. Coffee shops, airports, and hotels transformed from physical meeting places to universal hotspots, enabling the masses to bridge distances, connect instantly, and effectively trivialize time and space quite literally in the palms of their hands. This is the power of wireless technology.

So then, why does society continue to labor under the burden of wires? Having enjoyed rewards of another relatively nascent wireless technology, why do we hesitate to cut the cord?

Wireless power transfer (WPT)-enabled mobile devices have the potential to be a disruptive technology on the consumer technology market. Investments in WPT have grown dramatically in the past decade, and more technologies have started integrating WPT into their systems at the consumer level.

The result is a growing number of WPT methods seeing practical application. These techniques can successfully power and charge a wide range of consumer products, from cell phones to cars. However, they still face significant disadvantages in transmission distance and efficiency. Most require line of sight to operate.

Gemstone Team TESLA proposes time-reversed electromagnetic wave propagation (to be abbreviated ``time-reversal'' or ``TR'' in this document) as a novel WPT method. In particular, TR's ability to maintain connection even when line of sight is lost opens up a new range of possible applications. To support this proposal, the team has performed a number of exploratory experiments on the topic of TR applied to WPT.  These experiments are concerned primarily with understanding the fundamental engineering realities that would be necessary to build a working TR WPT system.

In this thesis, TESLA overviews the state of the art of both consumer wireless power transfer technologies and time reversal. Major innovations are outlined, as are current limitations. The team's experiments are then detailed individually. The purpose, methodology, and results for each experiment are explained. The results of each experiment and their significance are then discussed in the context of TR WPT. The results of the team as a whole are discussed in a similar way. Finally, the team makes suggestions for future research that could be made in this field, for the benefit of future researchers.