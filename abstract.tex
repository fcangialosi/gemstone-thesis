\hbox{\ }

\startSINGLEspacing

\begin{center}
\large{{ABSTRACT}} 

\vspace{3em} 

\end{center}
\hspace{-.15in}
\begin{tabular}{ll}
Title of Thesis:    & {\large TIME REVERSED ELECTROMAGNETIC  }\\
&				      				{\large WAVE PROPAGATION AS A NOVEL} \\
&				      				{\large METHOD OF WIRELESS POWER TRANSFER} \\
\ \\
&                     {\large Frank Cangialosi, Anu Challa, Tim Furman,} \\
&                     {\large Tyler Grover, Patrick Healey, Ben Philip,} \\
& 										{\large Scott Roman, Andrew Simon, Alex Tabatabai} \\
&                     {\large Liangcheng Tao} \\ 
\ \\
Thesis directed by: & {\large \mentor } \\
&  				{\large	\mentorsdepartment } \\
\end{tabular}

\vspace{3em}

\renewcommand{\baselinestretch}{2}
\large \normalsize

We investigate the application of time-reversed electromagnetic wave propagation
to transmit energy in a wireless power transmission system. ``Time reversal'' is
a signal focusing method that exploits the time reversal invariance of the lossless
wave equation to direct signals onto a single point inside a complex scattering
environment. In this work, we explore the properties of time reversed microwave 
pulses in a low-loss ray-chaotic chamber. We measure the spatial profile of the 
collapsing wavefront around the target antenna, and demonstrate that time reversal 
can be used to transfer energy to a receiver in motion. We demonstrate how nonlinear 
elements can be controlled to selectively focus on one target out of a group. 
Finally, we discuss the design of a rectenna for use in a time reversal system.
We explore the implication of these results, and how they may be applied in future
technologies.

\iffinal
	% no compile label
\else
	\par\noindent\centerline{\textbf{This version compiled on \today~at~\currenttime}}
\fi
